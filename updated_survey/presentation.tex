\documentclass[aspectratio=169]{beamer}
\usepackage[utf8]{inputenc}
\usepackage[T1]{fontenc}
\usepackage{lmodern}
\usepackage{graphicx}
\usepackage{booktabs}
\usepackage{amsmath,amssymb}
\usepackage{tikz}
\usepackage{algorithm}
\usepackage{algpseudocode}
\usepackage{listings}

% Graphics path
\graphicspath{{figures/}}

% Theme and colors
\usetheme{Madrid}
\usecolortheme{default}
\setbeamertemplate{navigation symbols}{}
\setbeamertemplate{footline}[frame number]

% Custom colors
\definecolor{darkblue}{RGB}{0,51,102}
\definecolor{lightblue}{RGB}{173,216,230}
\setbeamercolor{title}{fg=darkblue}
\setbeamercolor{frametitle}{fg=darkblue}
\setbeamercolor{structure}{fg=darkblue}

% Title information
\title[FPGA Vision Accelerators]{Vision FPGA Accelerators:\\A Comprehensive Survey}
\author{Mohammad Pivezhandi}
\institute{Wayne State University\\Department of Electrical and Computer Engineering}
\date{\today}

\begin{document}

%-------------------------------------------------------------------------------
% Title Slide
%-------------------------------------------------------------------------------
\begin{frame}
\titlepage
\end{frame}

%-------------------------------------------------------------------------------
% Outline
%-------------------------------------------------------------------------------
\begin{frame}{Outline}
\tableofcontents
\end{frame}

%-------------------------------------------------------------------------------
% Section 1: Introduction
%-------------------------------------------------------------------------------
\section{Introduction}

\begin{frame}{Motivation}
\begin{columns}
\begin{column}{0.5\textwidth}
\textbf{Problem Statement:}
\begin{itemize}
    \item Deep neural networks demand massive computational resources
    \item GPUs offer high throughput but consume significant power
    \item ASICs are efficient but lack flexibility
    \item Edge deployment requires low power and real-time processing
\end{itemize}
\end{column}
\begin{column}{0.5\textwidth}
\textbf{FPGA Advantages:}
\begin{itemize}
    \item Reconfigurable hardware
    \item Lower power than GPUs
    \item Customizable datapath
    \item Parallel processing capability
    \item Bridge between software and hardware
\end{itemize}
\end{column}
\end{columns}
\end{frame}

\begin{frame}{Survey Scope}
\textbf{This survey covers FPGA accelerators for:}
\vspace{0.5cm}
\begin{enumerate}
    \item \textbf{Feature Extraction Methods}
    \begin{itemize}
        \item Harris Corner Detection, SIFT, SURF
    \end{itemize}
    \item \textbf{Convolutional Neural Networks}
    \begin{itemize}
        \item LeNet, AlexNet, VGG, ResNet, MobileNet
    \end{itemize}
    \item \textbf{Vision Transformers}
    \begin{itemize}
        \item ViT, DeiT, Swin Transformer
    \end{itemize}
    \item \textbf{Spiking Neural Networks}
    \begin{itemize}
        \item Neuromorphic computing, Event-based vision
    \end{itemize}
    \item \textbf{Edge AI and TinyML}
    \begin{itemize}
        \item Low-power deployment, On-device learning
    \end{itemize}
\end{enumerate}
\end{frame}

%-------------------------------------------------------------------------------
% Section 2: Feature Extraction
%-------------------------------------------------------------------------------
\section{Feature Extraction Methods}

\begin{frame}{Feature Extraction on FPGAs}
\begin{columns}
\begin{column}{0.5\textwidth}
\textbf{Harris Corner Detection:}
\begin{itemize}
    \item Detects corners using autocorrelation
    \item Stream processing architecture
    \item Low dynamic power optimization
\end{itemize}

\textbf{SIFT Algorithm:}
\begin{itemize}
    \item Scale-invariant feature transform
    \item DoG pyramid construction
    \item Keypoint orientation assignment
\end{itemize}
\end{column}
\begin{column}{0.5\textwidth}
\textbf{SURF Algorithm:}
\begin{itemize}
    \item Speeded-up robust features
    \item Integral images for fast computation
    \item Hessian-based interest points
\end{itemize}

\textbf{FPGA Benefits:}
\begin{itemize}
    \item Real-time processing
    \item Parallel filter computation
    \item Pipeline architecture
\end{itemize}
\end{column}
\end{columns}
\end{frame}

%-------------------------------------------------------------------------------
% Section 3: CNNs on FPGAs
%-------------------------------------------------------------------------------
\section{CNN Accelerators}

\begin{frame}{CNN Architecture Overview}
\begin{center}
\includegraphics[width=0.7\textwidth]{deepfeatures.pdf}
\end{center}
\textbf{Key Components:}
\begin{itemize}
    \item Convolutional layers: Feature extraction with learned filters
    \item Pooling layers: Dimensionality reduction
    \item Fully connected layers: Classification
\end{itemize}
\end{frame}

\begin{frame}{FPGA Critical Parameters}
\begin{columns}
\begin{column}{0.5\textwidth}
\textbf{Data Type Organization:}
\begin{itemize}
    \item Fixed-point vs. floating-point
    \item Custom precision (6-bit exponent, 5-bit mantissa)
    \item Dynamic quantization per layer
\end{itemize}

\textbf{Storage Management:}
\begin{itemize}
    \item On-chip BRAM limitations
    \item Off-chip DRAM bandwidth
    \item Parameter encoding schemes
\end{itemize}
\end{column}
\begin{column}{0.5\textwidth}
\textbf{Bandwidth Optimization:}
\begin{itemize}
    \item Batch processing
    \item Weight reuse strategies
    \item Roofline model analysis
\end{itemize}

\textbf{Loop Optimization:}
\begin{itemize}
    \item Loop unrolling
    \item Loop tiling
    \item Loop interchange
\end{itemize}
\end{column}
\end{columns}
\end{frame}

\begin{frame}{Optimization Techniques}
\begin{columns}
\begin{column}{0.5\textwidth}
\textbf{Algorithmic Optimization:}
\begin{itemize}
    \item FFT-based convolution
    \item Winograd transform
    \item Layer fusion
\end{itemize}

\textbf{Quantization Methods:}
\begin{itemize}
    \item Dynamic fixed-point
    \item Binary Neural Networks (BNN)
    \item Incremental network quantization
\end{itemize}
\end{column}
\begin{column}{0.5\textwidth}
\textbf{Model Compression:}
\begin{itemize}
    \item Structured pruning
    \item Channel-wise pruning
    \item Sparse neural networks
\end{itemize}

\textbf{Design Automation:}
\begin{itemize}
    \item High-Level Synthesis (HLS)
    \item OpenCL frameworks
    \item Hardware-aware NAS
\end{itemize}
\end{column}
\end{columns}
\end{frame}

%-------------------------------------------------------------------------------
% Section 4: Vision Transformers
%-------------------------------------------------------------------------------
\section{Vision Transformers}

\begin{frame}{Vision Transformer Architecture}
\textbf{Key Components:}
\begin{itemize}
    \item \textbf{Patch Embedding:} Divide image into 16$\times$16 patches
    \item \textbf{Multi-Head Self-Attention (MHSA):} Compute attention between all patch pairs
    \item \textbf{Feed-Forward Network (FFN):} Two-layer MLP with GELU activation
    \item \textbf{Layer Normalization:} Applied before each sub-layer
\end{itemize}

\vspace{0.5cm}
\textbf{Computational Challenge:}
\begin{itemize}
    \item Attention complexity: $O(N^2 \cdot D)$ where $N$ = number of patches
    \item For 224$\times$224 image with 16$\times$16 patches: $N = 196$
\end{itemize}
\end{frame}

\begin{frame}{ViT FPGA Implementations}
\begin{table}
\centering
\small
\begin{tabular}{@{}lllrrr@{}}
\toprule
Work & FPGA & Model & TOPs/s & Power(W) & Eff. \\
\midrule
ViTA'21 & ZCU102 & ViT-B & 1.4 & 6.8 & 206 \\
AutoViT'22 & Stratix 10 & DeiT-S & 3.15 & 21.5 & 147 \\
FlightBERT'23 & Alveo U280 & ViT-B & 5.8 & 37.2 & 156 \\
SwinAcc'23 & Versal VCK190 & Swin-T & 4.7 & 38.3 & 123 \\
MobileViT'24 & ZCU104 & MobileViT-S & 0.89 & 3.2 & 278 \\
\bottomrule
\end{tabular}
\caption{FPGA Vision Transformer implementations comparison (Efficiency in GOPs/W)}
\end{table}
\end{frame}

%-------------------------------------------------------------------------------
% Section 5: Spiking Neural Networks
%-------------------------------------------------------------------------------
\section{Spiking Neural Networks}

\begin{frame}{Neuromorphic Computing}
\begin{columns}
\begin{column}{0.5\textwidth}
\textbf{Biological Inspiration:}
\begin{itemize}
    \item Mammalian cortex: 100 billion neurons
    \item Each neuron: 1,000--10,000 synapses
    \item Spike-based communication
    \item Graceful degradation
\end{itemize}

\textbf{Large-Scale Systems:}
\begin{itemize}
    \item IBM TrueNorth
    \item Intel Loihi
    \item Stanford Neurogrid
    \item Manchester SpiNNaker
\end{itemize}
\end{column}
\begin{column}{0.5\textwidth}
\textbf{SNN Advantages:}
\begin{itemize}
    \item Event-driven computation
    \item Energy efficiency
    \item Temporal processing
    \item Self-supervised learning (STDP)
\end{itemize}

\textbf{FPGA Benefits:}
\begin{itemize}
    \item Higher throughput than CPUs/GPUs
    \item Flexibility vs. ASICs
    \item Rapid prototyping
    \item Technology independence
\end{itemize}
\end{column}
\end{columns}
\end{frame}

\begin{frame}{Event-Based Vision}
\textbf{Dynamic Vision Sensors (DVS):}
\begin{itemize}
    \item Asynchronous pixel-level brightness change detection
    \item High temporal resolution ($\mu$s level)
    \item Low power consumption
    \item Natural fit for SNNs
\end{itemize}

\vspace{0.5cm}
\textbf{FPGA Applications:}
\begin{itemize}
    \item Real-time optical flow calculation
    \item Object detection and tracking
    \item Frequency extraction from rotating objects
    \item Histogram creation for event processing
\end{itemize}
\end{frame}

%-------------------------------------------------------------------------------
% Section 6: Edge AI
%-------------------------------------------------------------------------------
\section{Edge AI and TinyML}

\begin{frame}{Edge AI Requirements}
\textbf{Constraints for Edge Deployment:}
\begin{itemize}
    \item \textbf{Power Budget:} $<$5W for battery-powered, $<$15W for powered devices
    \item \textbf{Latency:} Real-time requirements (1--100ms)
    \item \textbf{Memory:} Limited on-chip BRAM (few MBs)
    \item \textbf{Cost:} Low-cost FPGAs (Zynq-7000, Artix-7, Cyclone V)
    \item \textbf{Model Size:} Compressed models $<$10MB
\end{itemize}

\vspace{0.5cm}
\textbf{Key Techniques:}
\begin{itemize}
    \item Sub-4-bit quantization
    \item Hardware-aware Neural Architecture Search
    \item On-device incremental learning
\end{itemize}
\end{frame}

\begin{frame}{Edge AI Applications}
\begin{table}
\centering
\small
\begin{tabular}{@{}lllrrr@{}}
\toprule
Application & FPGA & Model & FPS & Power(W) & Acc.(\%) \\
\midrule
ImageNet & PYNQ-Z2 & MobileNetV2 & 47 & 2.1 & 71.8 \\
Detection & ZCU104 & YOLOv5s & 42 & 6.8 & 44.2 \\
Face Rec. & Artix-7 & ArcFace & 89 & 1.4 & 99.3 \\
Segmentation & Zynq-7020 & U-Net & 37 & 2.8 & 94.1 \\
Re-ID & Cyclone V & OSNet-BNN & 156 & 0.9 & 91.2 \\
\bottomrule
\end{tabular}
\caption{Edge AI FPGA implementations across different application domains}
\end{table}
\end{frame}

%-------------------------------------------------------------------------------
% Section 7: Conclusion
%-------------------------------------------------------------------------------
\section{Conclusion}

\begin{frame}{Key Findings}
\begin{columns}
\begin{column}{0.5\textwidth}
\textbf{FPGA Advantages:}
\begin{itemize}
    \item High parallel processing capability
    \item Customizable datapaths
    \item Energy efficiency vs. GPUs
    \item Reconfigurability vs. ASICs
\end{itemize}

\textbf{Optimization Strategies:}
\begin{itemize}
    \item Quantization (INT8 to binary)
    \item Pruning and sparsity
    \item Algorithmic transforms (Winograd, FFT)
    \item Hardware-aware NAS
\end{itemize}
\end{column}
\begin{column}{0.5\textwidth}
\textbf{Emerging Trends:}
\begin{itemize}
    \item Vision Transformers on FPGAs
    \item Edge AI deployment
    \item On-device learning
    \item Event-based vision
\end{itemize}

\textbf{Future Directions:}
\begin{itemize}
    \item Neuromorphic computing
    \item Self-supervised learning
    \item Sub-mW vision systems
    \item Hybrid CNN-Transformer architectures
\end{itemize}
\end{column}
\end{columns}
\end{frame}

\begin{frame}{Conclusion}
\begin{center}
\Large
\textbf{FPGAs provide an optimal balance between:}

\vspace{0.5cm}
\begin{itemize}
    \item[$\bullet$] Performance and Power Efficiency
    \item[$\bullet$] Flexibility and Customization
    \item[$\bullet$] Real-time Processing and Accuracy
\end{itemize}

\vspace{1cm}
\textbf{Neuromorphic architectures represent the future of\\vision algorithms for real-time, energy-efficient processing.}
\end{center}
\end{frame}

%-------------------------------------------------------------------------------
% References
%-------------------------------------------------------------------------------
\begin{frame}{References}
\small
Key references available in the survey paper:\\
\textit{Vision FPGA Accelerators: A Comprehensive Survey}
\end{frame}

%-------------------------------------------------------------------------------
% Thank You
%-------------------------------------------------------------------------------
\begin{frame}
\begin{center}
\Huge Thank You!

\vspace{1cm}
\large Questions?

\vspace{1cm}
\normalsize
Mohammad Pivezhandi\\
Wayne State University\\
mpvzhndi@wayne.edu
\end{center}
\end{frame}

\end{document}
