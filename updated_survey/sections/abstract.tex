\section*{Abstract}

The encroachment of computers toward more sophisticated platforms and the availability of large scale databases makes deep neural networks prevailing. Despite the analog deep learning architectures emulate from biological neural systems behaviourally, they are not indisputably loyal to the brain functioning. As a result, with neuromorphic computing. we want to satisfy the insatiable desire for imitating from biological neural networks. This survey paper firstly elaborates the classic vision methods and then moves toward deep architectures\rev{, explores the emerging vision transformer architectures,} with the conclusion on neuromorphic computing as energy efficient future design.

All the vision algorithms, from classic methods to the learnable architectures, inspire from brain. The biological vision adds details to a coarse input like edges. The feature extraction methods, such as SIFT, HOG, and SURF, also extract corners and edges to create descriptors. Besides, the layers in deep learning models advance from simple to more complex in an hierarchical order. \rev{Recent developments in vision transformers introduce self-attention mechanisms that further enhance feature extraction capabilities.} Researchers focus on neuromorphic computing stems from spike based behavior of the mammalian cortex and the inclination toward being loyal to the biological implementations.

This article tries to give a comprehensive evaluation for computer vision FPGA accelerators \rev{covering research from 2015 to 2025}. We evaluate the critical parameters such as on-chip memory, communication bandwidth, and limited resources of FPGAs. The main gist of the covered publications is parallelization, memory and algorithmic optimizations, sparse connection, and network quantization\rev{, along with emerging techniques such as neural architecture search, dynamic precision scaling, and attention mechanism optimization}. \rev{This updated survey also covers vision transformers, edge AI implementations, and the latest FPGA platforms including Intel Agilex and AMD Versal.} In this evaluation, we want to conclude the neuromorphic deep learning architectures would be the future in vision algorithms in real-time processing and energy efficiency.
