\section*{Abstract}

\revb{The advancement of computing platforms and the availability of large-scale datasets have made deep neural networks increasingly prevalent.} \revb{Although conventional deep learning architectures emulate biological neural systems behaviorally, they do not faithfully replicate brain functioning.} \revb{Neuromorphic computing addresses this gap by more closely imitating biological neural networks.} \revb{This survey examines classic vision methods, progresses to deep architectures}\rev{, explores emerging vision transformer architectures,} \revb{and concludes with neuromorphic computing as an energy-efficient design paradigm.}

\revb{All vision algorithms, from classic methods to learnable architectures, draw inspiration from the brain.} \revb{Biological vision systems process coarse inputs by detecting features such as edges.} \revb{Feature extraction methods including SIFT, HOG, and SURF similarly extract corners and edges to create descriptors.} \revb{Deep learning models process features hierarchically, advancing from simple to complex representations across successive layers.} \rev{Recent developments in vision transformers introduce self-attention mechanisms that further enhance feature extraction capabilities.} \revb{The focus on neuromorphic computing stems from the spike-based behavior of the mammalian cortex and the goal of faithfully replicating biological implementations.}

\revb{This article provides a comprehensive evaluation of computer vision FPGA accelerators} \rev{covering research from 2015 to 2025}. \revb{We evaluate critical parameters including on-chip memory, communication bandwidth, and FPGA resource constraints.} \revb{The surveyed publications address parallelization, memory and algorithmic optimizations, sparse connections, and network quantization}\rev{, along with emerging techniques such as neural architecture search, dynamic precision scaling, and attention mechanism optimization}. \rev{This survey also covers vision transformers, edge AI implementations, and the latest FPGA platforms including Intel Agilex and AMD Versal.} \revb{We conclude that neuromorphic deep learning architectures represent the future of vision algorithms for real-time processing and energy efficiency.}
